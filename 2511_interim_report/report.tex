\documentclass[12pt,a4paper]{article}

\usepackage[T1]{fontenc}
\usepackage[utf8]{inputenc}
\usepackage{lmodern}

\usepackage[a4paper,margin=1in]{geometry}

\usepackage{xcolor}
\usepackage{graphicx}

\usepackage{array}
\usepackage{booktabs}
\usepackage{multirow}
\usepackage{tabularx}

\usepackage{fancyhdr}
\usepackage{lastpage}

\usepackage{setspace}
\onehalfspacing

\usepackage{enumitem}
\setlist[enumerate]{label=(\arabic*)}

\usepackage{hyperref}
\hypersetup{
    colorlinks=true,
    linkcolor=blue,
    urlcolor=blue,
    citecolor=blue,
    pdfnewwindow=true
}

\definecolor{headergray}{gray}{0.35}

\pagestyle{fancy}
\fancyhf{}

\newcommand{\HeaderCenterText}{%
  {\color{headergray}\sffamily\bfseries FYP Interim Report 2025-2026}\\
  {\color{headergray}\sffamily Department of Computer Science}%
}

\newcommand{\HeaderRightImage}{%
  \raisebox{-0.2\height}{\includegraphics[height=1.2cm]{mu.png}}%
}

\fancyhead[C]{\HeaderCenterText}
\fancyhead[R]{\HeaderRightImage}

\fancyfoot[C]{\sffamily Page \thepage\ of \pageref{LastPage}}

\renewcommand{\headrulewidth}{0pt}
\setlength{\headheight}{24pt}

\setlength{\parskip}{0.6em}
\setlength{\parindent}{0pt}

\begin{document}

\begin{table}[ht]
  \small
  \renewcommand{\arraystretch}{1.4}
  \setlength{\tabcolsep}{8pt}
  \begin{tabularx}{\textwidth}{@{} >{\bfseries}l X >{\bfseries}l X @{}}
    \toprule
    Student Name & David P. Callanan & Student Number & 21444104 \\
    Supervisor & Dr. Phil Maguire & ECTS Credits & 5 \\
    Project Title & \multicolumn{3}{p{0.75\textwidth}}{Studying link-time optimizations in programming language development to facilitate the continuum of static and dynamic modules.} \\
    \bottomrule
  \end{tabularx}
\end{table}

\section{Project Objectives}

The goal of this project is to develop a proof-of-concept programming language that facilitates bespoke optimization techniques via novel language constructs. The specific objective is to research the continuum of static and dynamic modules in low-level software development. A module system would be designed to promote dispatch flexibility by abstracting away the underlying dispatch mechanism, enabling aggressive link-time optimizations when circumstances allow. The development of a prototype compiler would demonstrate the efficacy of the proposed solution.


\section{Description of Work Completed}


\subsection{Evidence of Work Completed}


\subsection{Literature Review}


\subsection{Use of GenAI and Tools}

Two notable GenAI tools have been used to assist with the research and development of this project so far.

\begin{enumerate}

\item The first is \href{https://t3.chat}{t3.chat}, a web application that consolidates numerous large language models into a streamlined chat interface \cite{t3chat}. The user can experiment with different models to understand which models are best suited for the task at hand. Both research and programming tasks have benefited from the use of this tool. The tool was primarily used to steer the work in the right direction and to identify the concepts, libraries and resources through which further research could be conducted. For transparency, all LLM prompts have been included in \textbf{Appendix B}.

\item The second is the tab-complete feature of GitHub Copilot, which 

\end{enumerate}


\section{Future Work}



\begin{thebibliography}{9}
\bibitem{t3chat} ``Meet T3 Chat AI: Your All-in-One, Super-Fast AI Assistant,'' DigiVirus. \url{https://digivirus.in/meet-t3-chat-ai-your-all-in-one-super-fast-ai-assistant/}
\end{thebibliography}

\appendix

\newpage

\section{Appendix: Source Code}


\newpage

\section{Appendix: LLM Prompts}


\label{LastPage}
\end{document}
